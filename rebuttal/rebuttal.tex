\documentclass[11pt]{article}
% Packages ------------------------------------------------------------------
	\usepackage[utf8]{inputenc}
	\usepackage{fullpage}
	% package needed for optional arguments
	\usepackage{xifthen}
	% define counters for reviewers and their points
	\newcounter{reviewer}
	\setcounter{reviewer}{0}
	\newcounter{point}[reviewer]
	\setcounter{point}{0}

	% This refines the format of how the reviewer/point reference will appear.
	\renewcommand{\thepoint}{\,\thereviewer.\arabic{point}}

	% command declarations for reviewer points and our responses
	\newcommand{\reviewersection}{\stepcounter{reviewer} \bigskip \hrule
	                  \section*{Reviewer \thereviewer}}


	\newenvironment{point}
	   {\refstepcounter{point} \bigskip \noindent \emph{\textbf{Reviewer~Point~\thepoint} } ---\ \itshape}
	   {\par }

	\newenvironment{reply}
	   {\medskip \noindent
	   	\textbf{Reply}:\  }
	   {\medskip }
\begin{document}
\section*{Response to the reviewers}
	% General intro text goes here
	We thank the reviewers for their critical assessment of our work.
	In the following we address their concerns point by point.

	Please note that reviewers' comments are in italics while our
	answers are using normal fonts.

% Let's start point-by-point with Reviewer 1 --------------------------
	\reviewersection

	The authors thank the reviewer for the valuable comments and suggestions; and we agree almost 100\% with that.
	\subsection*{General comments}

	I have to reject the paper in its present form.
	I suggest you to submit a rigorously revised manuscript taking these critics into account.


	\subsection*{Issues}
	\begin{point}
	 	\textbf{[Point heading]}It was difficult to identify the new aspects of your manuscript compared to:
	 	- former submission to EFM: EFM-D-21-00395
		- your publication in Theor. Appl. Fracture Mechanics 2020 (your reference [14]).
	\end{point}

	\begin{reply}
		Earlier we have submitted our manuscript to EFM: EFM-D-21-00395, considering the reviewer's suggestion we have modified the manuscript and also changed the title of manuscript and resubmitted to the journal EFM.

		Publication in Theor. Appl. Fracture Mechanics 2020 (your reference [14]) and this manuscript is very different. Though both of these manuscripts main aim is to give a computationally efficient phase-field method to solve brittle fracture problems.

		Reference [14] main  aim is to auto adaptively adjust the displacement step size while solving the coupled system of non- linear equations arising from the phase field formulation of brittle fracture.

		Whereas in the case of the present manuscript the main aim is to propose a new error-indicator based on the crack driving force to carry out adaptive mesh refinement of the domain for phase-field simulation of brittle fracture.

	\end{reply}
	\begin{point}
	 	\textbf{[Point heading]}Moreover, the plagiarism check found 16$\%$ overlap with this paper (your 	reference [14]), mainly in the introduction and in fundamental chapters.
	\end{point}

	\begin{reply}
		I am aware of the fact that the plagiarism check found 16 $\%$ overlap with my reference [14]. As both the papers aim is to propose a computationally efficient phase-field fracture method and the authors of both the paper are same. I can definately assure that both the manuscript have their own novelty.
	\end{reply}

	\subsection*{Minor Issues}

	\begin{point}
	 	\textbf{[Point heading]} Please include a nomenclature in alphabetic order: Latin, Greek, Acronyms.
	\end{point}

	\begin{reply}
		We have included nomenclature to the manuscript.
	\end{reply}

	\begin{point}
	 	\textbf{[Point heading]} Each HIGHLIGHT must not be longer than 85 characters – Please shorten them.
	\end{point}

	\begin{reply}
		We have modified the highlights.
	\end{reply}

	\begin{point}
	 	\textbf{[Point heading]} Number of references should be shortened to essential ones.
	\end{point}

	\begin{reply}
		This point is also noted.
	\end{reply}


\end{document}



